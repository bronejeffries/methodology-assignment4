\documentclass[11pt]{article}

\usepackage{cite}
\begin{document}

\title{\textbf{MAKERERE UNIVERSITY\\COLLEGE OF COMPUTING AND INFORMATICS\\}}

\author{Ssemate Brian\\216000206\\16/U/1134}
\renewcommand{\today}{}

\maketitle

\newpage


\textbf{ANDROID OPERATING SYSTEM.}
\section*{INTRODUCTION}

Among the amazing products that have ever been opened to the world, the android operating system is recognized. Android is a  mobile operating system developed by Google, based on a modified version of the Linux kernel and other open source software and designed primarily for touchscreen mobile devices such as smart phones and tablets.\cite{s1} In addition, Google has further developed Android TV for televisions, Android Auto for cars, and Android Wear for wrist watches, each with a specialized user interface. Variants of Android are also used on game consoles, digital cameras, PCs and other electronics.
Initially developed by Android Inc., which Google bought in 2005, Android was unveiled in 2007, with the first commercial Android device launched in September 2008. The operating system has since gone through multiple major releases, with the current version being 8.1 "Oreo", released in December 2017.


\section*{OVERVIEW}

Before android the leading mobile operating systems were Symbian and Microsoft Windows Mobile.\cite{s2}\cite{s3}  
Android has been the best-selling OS worldwide on smart phones since 2011 and on tablets since 2013. As of May 2017, it has over two billion monthly active users, the largest installed base of any operating system, and as of 2017, the Google Play store features over 3.5 million apps.\cite{s4}

\section*{FUNCTIONALITY}
Android's default user interface is mainly based on direct manipulation, using touch inputs that loosely correspond to real-world actions, like swiping, tapping, pinching, and reverse pinching to manipulate on-screen objects, along with a virtual keyboard.The response to user input is designed to be immediate and provides a fluid touch interface, often using the vibration capabilities of the device to provide haptic feedback to the user.\cite{s5}
\bibliographystyle{IEEEtran}
\bibliography{DB}
\end{document}